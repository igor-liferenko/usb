\catcode`\"=13 \def"{\leavevmode\hbox\bgroup\let"=\egroup\def\do##1{\catcode`##1=12}\dospecials\tt}
\nopagenumbers

\newcount\lineno
\def\setupverbatim{\tt
  \obeylines
  \def\do##1{\catcode`##1=12 } \dospecials
  \obeyspaces}
{\obeyspaces\global\let =\ }
\def\listing#1{\par\begingroup\setupverbatim\input#1 \endgroup}
{
  \everypar={\advance\lineno by 1 \ifnum\lineno<60 \llap{\sevenrm \the\lineno\quad}\fi}
  \listing{stest.txt}
}
\bigskip
\noindent{\bf Overview:}\par
\item{0.} An implicit space can be an active character (which is called ``active'' here; we must
separate ``active space'' from ``active space-token'') or
a control sequence.
\item{1.} In "\ss" we branch catcode 10 or not.
\item{2.} In "\testactive" we branch active or implicit.
\item{3.} Parameter text for "\sss" is "#1 #2" because: if first token in token list is explicit
space, it is matched to the space in parameter text and hence "#1" will be empty. On the other
hand,
if the first token
in token list is implicit space or active space, an explicit space in the token list (if there is one --- not in the first position) or the space after the "!" (if there is no explicit space in token list) is matched to the space in parameter text, and "#1" will include all tokens in the stream before the explicit space.
\item{4.} Each space token (if it is the first in token list) is handled via two separate routes.
In the first route
(via "\futurelet", which makes "\next" an implicit space token, regardless whether the space token is explicit/implicit/active) we detect catcode 10 in line 10.
In the second route (via argument "#1" to macros "\sss" and "\ssss", which leaves each
space token intact) we detect
explicit/implicit/active.
\item{5.} In line 37 we branch between active and explicit. There is a subtlety: we use "\uppercase" to turn the active character from implicit space into a macro, to prevent substituting the active character with what is inside of it, during the "\ifcat" test ("~" is a macro by default).
\item{6.} For "ACTIVE" there are two cases:
\itemitem{[a]} "+" $\rightarrow$ normal
\itemitem{[b]} "+" $\rightarrow$ funny
\item{} There are two more special cases:
\itemitem{[c]} charcode of active character is 32 (this case may be split into two sub-cases: when
the space, that is inside the active character, has charcode 32; and when charcode is not 32).
\itemitem{[d]} charcode of active character and charcode, which is inside of the active character, are the same (they must not be equal to 32 --- this is a sub-case of [c]; and they must not be equal to 126 --- it is used in line 37).
\bigskip
\noindent\hrulefill\quad 0. Output is empty.\quad \hrulefill\null
\smallskip
\listing{stest0.tex}
\medskip
In line 8 "\next" becomes equivalent to "!". In line 10 "\ifcat" is false.
In line 46 "#1" is "! ".
\bigskip
\noindent\hrulefill\quad 1. Output is {\tt SPACE EXPLICIT}\quad \hrulefill\null
\smallskip
\listing{stest1.tex}
\medskip
...
\bigskip
\noindent\hrulefill\quad 2. Output is {\tt SPACE FUNNY EXPLICIT}\quad \hrulefill\null
\smallskip
\listing{stest2.tex}
\medskip
...
\bigskip
\noindent\hrulefill\quad 3. Output is {\tt SPACE}\quad \hrulefill\null
\smallskip
\listing{stest3.tex}
\medskip
...
\bigskip
\noindent\hrulefill\quad 4. Output is {\tt SPACE FUNNY}\quad \hrulefill\null
\smallskip
\listing{stest4.tex}
\medskip
...
\bigskip
\noindent\hrulefill\quad 5. Output is {\tt SPACE ACTIVE}\quad \hrulefill\null
\smallskip
\listing{stest5.tex}
\medskip
"\testactive"${}_1$ empty
\bigskip
\noindent\hrulefill\quad 6. Output is {\tt SPACE ACTIVE}\quad \hrulefill\null
\smallskip
\listing{stest6.tex}
\medskip
"\testactive"${}_2$ empty
\bigskip
\noindent\hrulefill\quad 7. Output is {\tt SPACE ACTIVE}\quad \hrulefill\null
\smallskip
\listing{stest7.tex}
\medskip
"\testactive"${}_1$ non-empty
\bigskip
\noindent\hrulefill\quad 8. Output is {\tt SPACE FUNNY ACTIVE}\quad \hrulefill\null
\smallskip
\listing{stest8.tex}
\medskip
"\testactive"${}_2$ non-empty
\bigskip
\noindent\hrulefill\quad 9. Output is {\tt SPACE FUNNY ACTIVE}\quad \hrulefill\null
\smallskip
\listing{stest9.tex}
\medskip
"\uppercase"
\bye
