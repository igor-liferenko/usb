\catcode`\"=13 \def"{\leavevmode\hbox\bgroup\let"=\egroup\def\do##1{\catcode`##1=12}\dospecials\tt}
\nopagenumbers

\newcount\lineno
\def\setupverbatim{\tt
  \obeylines
  \def\do##1{\catcode`##1=12 } \dospecials
  \obeyspaces}
{\obeyspaces\global\let =\ }
\def\listing#1{\par\begingroup\setupverbatim\input#1 \endgroup}
{
  \everypar={\advance\lineno by 1 \ifnum\lineno<60 \llap{\sevenrm \the\lineno\quad}\fi}
  \listing{stest.h}
}
\bigskip
\noindent{\bf Overview:}\par
\item{$\bullet$} An implicit space can be an active character (which is called ``active'' here; we must
separate ``active space'' from ``active space-token'') or
a control sequence.
\item{$\bullet$} In "\ss" we branch catcode 10 or not.
\item{$\bullet$} In "\sx" we branch beween normal and funny.
\item{$\bullet$} In "\sss" we branch between explicit and active/implicit.
\item{$\bullet$} In "\ssss" we branch between active/explicit and active/implicit.
\item{$\bullet$} In "\testactive" we branch between active and implicit.
\item{$\bullet$} Parameter text for "\sss" is "#1 #2" because: if first token in token list is explicit
space, it is matched to the space in parameter text and hence "#1" will be empty. On the other
hand,
if the first token
in token list is implicit space or active space, an explicit space in the token list (if there is one --- not in the first position) or the space after the "!" (if there is no explicit space in token list) is matched to the space in parameter text, and "#1" will include all tokens in the stream before the explicit space.
\item{$\bullet$} Each space token (if it is the first in token list) is handled via two separate routes.
In the first route
(via "\futurelet", which makes "\next" an implicit space token, regardless whether the space token is explicit/implicit/active) we detect catcode 10 in line 10.
In the second route (via argument "#1" to macros "\sss" and "\ssss", which leaves each
space token intact) we detect
explicit/implicit/active.
\item{$\bullet$} In line 37 we branch between active and explicit. There is a subtlety: we use "\uppercase" to turn the active character from implicit space into a macro, to prevent substituting the active character with what is inside of it, during the "\ifcat" test ("~" is a macro by default).
\item{$\bullet$} For "ACTIVE" there are two cases (let's assume that active character is `"+"'):
\itemitem{[a]} "+" $\rightarrow$ normal
\itemitem{[b]} "+" $\rightarrow$ funny
\item{} There are two more special cases:
\itemitem{[c]} charcode of active character is 32 (this case may be split into two sub-cases: when
the space, that is inside the active character, has charcode 32; and when charcode is not 32).
\itemitem{[d]} charcode of active character and charcode, which is inside of the active character, are the same (they must not be equal to 32 --- this is a sub-case of [c]; and they must not be equal to 126 --- it is used in line 37).
\item{$\bullet$} Argument list for "\tact" is formed by
running "\string" on the first token in token list (it is assumed that "\escapechar" is between 0 and 255). If this token is not active, there will be two or more tokens in arument list. If this token is an active character with
charcode not 32, there will be one token in argument list; if this token is an active character with charcode 32, argument list will be empty. "\tact" detects these two cases in order to determine whether the token is active.
\item{$\bullet$} We change "\escapechar" in line 34 in order that it cannot be equal
to charcode of
space (explicit or inside of active and implicit) --- to guarantee that test in line 35 cannot
be true for control sequences.
\item{$\bullet$} "\long" is used in case token list contains "\par".
\vfil\eject
\noindent\hrulefill\quad 0. Output is empty.\quad \hrulefill\null
\smallskip
\listing{stest0.tex}
\medskip
\noindent "\next" is equivalent to "!". Test in line 10 is false.
In line 46 "#1" is "! ".
\bigbreak
\noindent\hrulefill\quad 1. Output is {\tt SPACE EXPLICIT}\quad \hrulefill\null
\smallskip
\listing{stest1.tex}
\medskip
\noindent We get into "\sss". "#1" is empty, "#2" is "! ". Test in line 26 is true.
\bigbreak
\noindent\hrulefill\quad 2. Output is {\tt SPACE FUNNY EXPLICIT}\quad \hrulefill\null
\smallskip
\listing{stest2.tex}
\medskip
\noindent We get into "\ssss". "#1" is "*", "#2" is "! ". Test in line 35 is true.
Test in line 37 is false.
\bigbreak
\noindent\hrulefill\quad 3. Output is {\tt SPACE}\quad \hrulefill\null
\smallskip
\listing{stest3.tex}
\medskip
\noindent We get into "\sss". "#1" is "\stoken!", "#2" is empty. Test in line 26 is false.
We get into "\testactive". "#1" is "\stoken", "#2" is "!".
We get into "\tact". "#1" is "\", "#2" is "stoken\s".
\bigbreak
\noindent\hrulefill\quad 4. Output is {\tt SPACE FUNNY}\quad \hrulefill\null
\smallskip
\listing{stest4.tex}
\medskip
\noindent We get into "\ssss". "#1" is "\ftoken", "#2" is "! ". Test in line 35 is false.
We get into "\testactive". "#1" is "\ftoken", "#2" is empty.
We get into "\tact". "#1" is "\", "#2" is "ftoken\s".
\bigbreak
\noindent\hrulefill\quad 5. Output is {\tt SPACE ACTIVE}\quad \hrulefill\null
\smallskip
\listing{stest5.tex}
\medskip
\noindent We get into "\sss". "#1" is " !", "#2" is empty. Test in line 26 is false.
We get into "\testactive". "#1" is " ", "#2" is "!".
We get into "\tact". "#1" is "\s", "#2" is empty. Test in line 52 is false.
Test in line 55 is true.
\bigbreak
\noindent\hrulefill\quad 6. Output is {\tt SPACE ACTIVE}\quad \hrulefill\null
\smallskip
\listing{stest6.tex}
\medskip
\noindent We get into "\ssss". "#1" is " ", "#2" is "! ". Test in line 35 is false.
We get into "\testactive". "#1" is " ", "#2" is empty.
We get into "\tact". "#1" is "\s", "#2" is empty. Test in line 52 is false.
Test in line 55 is true.
\bigbreak
\noindent\hrulefill\quad 7. Output is {\tt SPACE ACTIVE}\quad \hrulefill\null
\smallskip
\listing{stest7.tex}
\medskip
\noindent We get into "\sss". "#1" is "+!", "#2" is empty. Test in line 26 is false.
We get into "\testactive". "#1" is "+", "#2" is "!".
We get into "\tact". "#1" is "+", "#2" is "\s". Test in line 52 is true.
\bigbreak
\noindent\hrulefill\quad 8. Output is {\tt SPACE FUNNY ACTIVE}\quad \hrulefill\null
\smallskip
\listing{stest8.tex}
\medskip
\noindent We get into "\ssss". "#1" is "+", "#2" is "! ". Test in line 35 is false.
We get into "\testactive". "#1" is "+", "#2" is empty.
We get into "\tact". "#1" is "+", "#2" is "\s". Test in line 52 is true.
\bigbreak
\noindent\hrulefill\quad 9. Output is {\tt SPACE FUNNY ACTIVE}\quad \hrulefill\null
\smallskip
\listing{stest9.tex}
\medskip
\noindent We get into "\ssss". "#1" is "*", "#2" is "! ". Test in line 35 is true.
Test in line 37 is true.
\bye
